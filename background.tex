% $Id: conclusion.tex 
% !TEX root = main.tex

%%
\section{Background}
\label{sec:background}

In \ac{RL}, intelligent agents learn to map environmental situations (environment states) to actions to maximize a numerical reward signal they receive from the environment, in the long term~\cite{sutton18}. \ac{RL} agents are defined by
$\mathcal{S}$: the state space, consisting of all relevant states of the environment,
$\mathcal{A}$: the action space, i.e. the set of all actions an agent can perform that affect the environment, and the reward $r$: the numerical signal encoding the positive or negative impact of the action at each execution step.

Q learning~\cite{watkins92} is a model-free implementation of the widely used model
is \ac{RL}. The long-term quality of an action performed at a given state is computed iteratively in a series of steps and is represented by a Q value,
$\mathit{Q(s,a)}$.
Formally, each execution step $t$ captures information from the environment and maps it to a state
{\color{purple}$s_t$} $\in \mathcal{S}$ in its state space. It then selects an
action {\color{purple}$a_t$} $\in \mathcal{A}$ from its action space and executes it. The agent
receives a reward {\color{Bittersweet}$r_t$} from the environment when it moves to the next state
$s_{t+1} \in \mathcal{S}$. The reward is used to update the optimality of performing the
action {\color{Mulberry}$a_t$} at state {\color{purple}$s_t$}. The agent's goal is to learn
a policy (\i.e. the best-fit action for each state) that maximizes the reward of the
long-run behavior. The learning rate \lrate{\alpha} determines how much new experiences 
overwrite previously learned experiences, and the discount factor {\color{RoyalBlue} $\gamma$} 
determines how much future rewards are discounted so that agents prioritize immediate actions and 
can plan the best long-term actions. At each time step $t$, the Q value of an action 
{\color{purple}$a_{t+1}$} taken in state {\color{purple}$s_{t+1}$}, $Q(s_{t+1}, a_{t+1})$, is updated by 
the Bellman learning equation as follows:

\vspace{1em}

\begin{equation*} \label{eq:QL}
     {\tikzmarknode{qt}{\highlight{purple}{$Q(s_t, a_t)$}}} +
    {\tikzmarknode{alpha}{\highlight{NavyBlue}{$\alpha$}}}
    [ 
    {\tikzmarknode{r}{\highlight{Bittersweet}{$r_{t+1}$}}}  +  
    {\tikzmarknode{gamma}{\highlight{RoyalBlue}{$\gamma$}}}
    {\tikzmarknode{max}{\highlight{OliveGreen}{$\max\limits_a Q(s_{t+1},a)$}}} - 
    {\tikzmarknode{qt2}{\highlight{purple}{$Q(s_t, a_t)$}}} 
    ]
\end{equation*}

\begin{tikzpicture}[overlay,remember picture,>=stealth,nodes={align=left,inner ysep=1pt},<-]
    % For "Qt1"
    \path (qt.north) ++ (3.9,1.7em) node[anchor=south east,color=Mulberry!85] (ntext){\textsf{\footnotesize Q-value}};
    \draw [color=Mulberry](qt.north) |- ([xshift=0.8ex,color=Mulberry]ntext.south west);
    \path (qt2.north) ++ (-2.2,1.8em) node[anchor=south east,color=Mulberry!85] (qt2text){};
    \draw [color=Mulberry](qt2.north) |- ([xshift=-4.9ex,color=Mulberry]qt2text.south west);
    % For alpha
    \path (alpha.north) ++ (-0.2,-2.8em) node[anchor=south east,color=NavyBlue] (atext){\textsf{\footnotesize learning rate}};
    \draw [color=NavyBlue](alpha.south) |- ([xshift=-9.3ex,color=NavyBlue]atext.south east);
    % For r
    \path (r.north) ++ (-0.1,1.5em) node[anchor=north east,color=Bittersweet!85] (lijtext){\textsf{\footnotesize reward}};
    \draw [color=Bittersweet](r.north) |- ([xshift=-4.3ex,color=Bittersweet]lijtext.south east);
    %gamma
    \path (gamma.north) ++ (0.5,1.5em) node[anchor=north west,color=RoyalBlue!85] (gtext){\textsf{\footnotesize discount factor}};
    \draw [color=RoyalBlue](gamma.north) |- ([xshift=-2.9ex,color=RoyalBlue]gtext.south east);
    % For "l_i^max"
    \path (max.north) ++ (-1.2,-3.6em) node[anchor=south west,color=xkcdHunterGreen!85] (lmaxtext){\textsf{\scriptsize Maximum Q-Value in the next state}};
    \draw [color=xkcdHunterGreen](max.south) |- ([xshift=-5ex,color=xkcdHunterGreen]lmaxtext.north);
\end{tikzpicture}




\endinput

\hfil

\begin{minipage}{0.5\columnwidth}
\begin{equation*}
    \label{eq:ab_crypto}
    \hspace*{-6em}
    X_{i} = \frac{1}{\sum_{i=1}^{\tikzmarknode{n}{\highlight{purple}{N}}} 
    \sum_{j=1}^{\tikzmarknode{mi}{\highlight{blue}{$M_i$}}} 
    \tfrac{\tikzmarknode{lij}{\highlight{Bittersweet}{$l_i^j$}}}{\tikzmarknode{lmax}{\highlight{OliveGreen}{$l^{max}$}}}
    }
\end{equation*}
\vspace*{0.8\baselineskip}
\begin{tikzpicture}[overlay,remember picture,>=stealth,nodes={align=left,inner ysep=1pt},<-]
    % For "N"
    \path (n.north) ++ (0,1.8em) node[anchor=south east,color=Plum!85] (ntext){\textsf{\footnotesize number of objects}};
    \draw [color=Plum](n.north) |- ([xshift=-0.3ex,color=Plum]ntext.south west);
    % For "M_i"
    \path (mi.north) ++ (0,3.5em) node[anchor=north west,color=blue!85] (mitext){\textsf{\footnotesize number of other objects}};
    \draw [color=blue](mi.north) |- ([xshift=-0.3ex,color=blue]mitext.south east);
    % For "l_i^j"
    \path (lij.north) ++ (0,1.9em) node[anchor=north west,color=Bittersweet!85] (lijtext){\textsf{\footnotesize size of j\textsuperscript{th} service}};
    \draw [color=Bittersweet](lij.north) |- ([xshift=-0.3ex,color=Bittersweet]lijtext.south east);
    % For "l_i^max"
    \path (lmax.north) ++ (-2.7,-1.5em) node[anchor=north west,color=xkcdHunterGreen!85] (lmaxtext){\textsf{\footnotesize maximum obj size}};
    \draw [color=xkcdHunterGreen](lmax.south) |- ([xshift=-0.3ex,color=xkcdHunterGreen]lmaxtext.south west);
\end{tikzpicture}
\end{minipage}
\caption{Two Equations side-by-side using minipage and figure constructs.}



\endinput

