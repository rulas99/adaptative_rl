% $Id: introduction.tex $
% !TEX root = main.tex

\section{Introduction}
\label{sec:introduction}

\acf{RL} is a subset of \ac{ML} techniques in which agents learn by interacting with an environment, 
through trial and error. An agent receives a scalar reward (positive or negative) for each interaction 
step, based on the action taken. The objective of the agent is to maximize its cumulative long-term 
reward~\cite{sutton98}. Traditional \ac{RL} algorithms, such as Q-Learning, assume stationary 
\acp{MDP}, where transition probabilities and reward functions are constant over 
time~\cite{meta-rl-traffic}. However, this assumption restricts \ac{RL} agents' effectiveness in 
real-world environments, which often exhibit non-stationary characteristics like evolving state 
distributions, varying reward dynamics, and changing action 
spaces~\cite{khetarpal2022continualreinforcementlearningreview}.

To overcome these limitations, our work adopts a \acf{CRL} paradigm, where agents continuously 
adapt to environmental changes~\cite{abel2023definitioncontinualreinforcementlearning}. Approaches 
such as meta-learning~\cite{zintgraf21} and transfer learning~\cite{zhuang20} have shown promise 
within \ac{CRL}. Meta-learning allows an agent to \emph{learn how to learn}, enhancing adaptability in 
new contexts~\cite{beck2024surveymetareinforcementlearning}. Transfer learning leverages prior 
knowledge to accelerate learning in related tasks~\cite{chen2022transferredqlearning}.

This paper introduces \adaptiverl, a Q-learning-based \ac{RL} algorithm that incorporates adaptive 
mechanisms inspired by \ac{CRL} to account for environment changes in the goal/rewards definition 
or in the action space available to an agent. Our work is motivated by the suitability of \ac{RL} for 
self-adaptive systems, given its capacity to dynamically adjust behavior in response to environmental 
feedback~\cite{HENRICHS2022106940}. However, real-world scenarios such as traffic control 
present substantial challenges due to inherent non-stationary dynamics~\cite{meta-rl-traffic}. 
Moreover, the problem of environment changes to the state space have been 
addressed~\cite{gueriau19}, while adaptations to the goals or action space remain an open question.

Our method effectively addresses non-stationary environments by continuously updating its learning strategy. 

We validate \adaptiverl using a self-adaptive systems context, specifically applied to traffic signal 
control at city intersections. In our scenario signal phases continuously influence state transitions 
and rewards~\cite{meta-rl-traffic}. By demonstrating the effectiveness of \adaptiverl under these conditions, we underline its suitability for self-adaptive systems, particularly those relying on streaming data where concept drift may occur, making efficient adaptation essential.

%%possibly remove
This paper is structured as follows: \fref{sec:background} introduces theoretical foundations relevant to our implementation. \fref{sec:related} reviews related work. \fref{sec:implementation} presents the \adaptiverl algorithm. \fref{sec:validation} describes our validation approach in traffic signal control and discusses results. Finally, \fref{sec:conclusion} offers conclusions and future research directions.





\endinput

