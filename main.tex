% !TeX document-id = {4951685f-4d1a-4629-8de6-69a05c8522c0}
%  $Id: main.tex $
% !BIB TS-program = bibtex

\RequirePackage{graphicx}
\documentclass[10pt, conference, ]{IEEEtran}
\IEEEoverridecommandlockouts
% The preceding line is only needed to identify funding in the first footnote. If that is unneeded, please comment it out.
% $Id: preamble.tex 3335 2012-10-31 10:43:42Z nicolas.cardozo $

%----[ Packages ]---

\usepackage{ifdraft}

\usepackage[T1]{fontenc}
\usepackage[utf8]{inputenc}
\usepackage{amsmath}
\usepackage{mathrsfs}
\usepackage{hyperref}
\usepackage{wasysym}
%\usepackage{pxfonts}
\usepackage{yfonts}
\usepackage[plain]{fancyref}
\usepackage{subcaption}
\let\labelindent\relax
\usepackage[final]{listings}
\usepackage{booktabs}
\usepackage{multirow}
%\usepackage{threeparttable}
\usepackage{xcolor,colortbl}
\usepackage{xspace}
\usepackage{suffix}
\usepackage{etoolbox}
\usepackage[inline]{enumitem}
\usepackage{acronym}
\usepackage{url}
\usepackage{textgreek} % Para caracteres griegos
\usepackage{algorithm}
\usepackage{algpseudocode}

% Package 'biblatex'
\usepackage[backend=bibtex,
   style=ieee,
   citestyle=numeric-comp,
   natbib=true,
   maxnames=7,
   minnames=1,
   maxcitenames=1, 
   mincitenames=1,
   giveninits=true,
   hyperref=true,
   sorting=none,
   defernumbers]{biblatex}

%\renewcommand{\bibfont}{\footnotesize}

\addbibresource{local.bib}


% fancyref
\def\fref{\Fref} % treat all \frefs as \Frefs
\renewcommand{\lstlistingname}{Snippet}
\newcommand*{\fancyreflstlabelprefix}{lst} % define lst delimiter
\newcommand*{\Freflstname}{\lstlistingname}
\newcommand*{\freflstname}{\lstlistingname}
\Frefformat{vario}{\fancyreflstlabelprefix}%
  {\Freflstname\fancyrefdefaultspacing#1#3}
\frefformat{vario}{\fancyreflstlabelprefix}%
  {\freflstname\fancyrefdefaultspacing#1#3}
\Frefformat{plain}{\fancyreflstlabelprefix}%
  {\Freflstname\fancyrefdefaultspacing#1}
\frefformat{plain}{\fancyreflstlabelprefix}%
  {\freflstname\fancyrefdefaultspacing#1}

\newcommand*{\fancyrefthmlabelprefix}{thm} % define thm delimiter
\newcommand*{\Frefthmname}{Theorem}%
\newcommand*{\frefthmname}{%
 \MakeLowercase{\Frefthmname}}%
\Frefformat{vario}{\fancyrefthmlabelprefix}%
  {\Frefthmname\fancyrefdefaultspacing#1#3}
\frefformat{vario}{\fancyrefthmlabelprefix}%
  {\frefthmname\fancyrefdefaultspacing#1#3}
\Frefformat{plain}{\fancyrefthmlabelprefix}%
  {\Frefthmname\fancyrefdefaultspacing#1}
\frefformat{plain}{\fancyrefthmlabelprefix}%
  {\frefthmname\fancyrefdefaultspacing#1}

\newcommand*{\fancyreflemlabelprefix}{lem} % define lem delimiter
\newcommand*{\Freflemname}{Lemma}%
\newcommand*{\freflemname}{%
 \MakeLowercase{\Freflemname}}%
\Frefformat{vario}{\fancyreflemlabelprefix}%
  {\Freflemname\fancyrefdefaultspacing#1#3}
\frefformat{vario}{\fancyreflemlabelprefix}%
  {\freflemname\fancyrefdefaultspacing#1#3}
\Frefformat{plain}{\fancyreflemlabelprefix}%
  {\Freflemname\fancyrefdefaultspacing#1}
\frefformat{plain}{\fancyreflemlabelprefix}%
  {\freflemname\fancyrefdefaultspacing#1}

\newcommand*{\fancyrefdeflabelprefix}{def} % define def delimiter
\newcommand*{\Frefdefname}{Definition}%
\newcommand*{\frefdefname}{%
 \MakeLowercase{\Frefdefname}}%
\Frefformat{vario}{\fancyrefdeflabelprefix}%
  {\Frefdefname\fancyrefdefaultspacing#1#3}
\frefformat{vario}{\fancyrefdeflabelprefix}%
  {\frefdefname\fancyrefdefaultspacing#1#3}
\Frefformat{plain}{\fancyrefdeflabelprefix}%
  {\Frefdefname\fancyrefdefaultspacing#1}
\frefformat{plain}{\fancyrefdeflabelprefix}%
  {\frefdefname\fancyrefdefaultspacing#1}

\newcommand*{\fancyreflnlabelprefix}{ln} % define ln delimiter
\newcommand*{\Freflnname}{Line}%
\newcommand*{\freflnname}{%
 \MakeLowercase{\Freflnname}}%
\Frefformat{vario}{\fancyreflnlabelprefix}%
  {\Freflnname\fancyrefdefaultspacing#1#3}
\frefformat{vario}{\fancyreflnlabelprefix}%
  {\freflnname\fancyrefdefaultspacing#1#3}
\Frefformat{plain}{\fancyreflnlabelprefix}%
  {\Freflnname\fancyrefdefaultspacing#1}
\frefformat{plain}{\fancyreflnlabelprefix}%
  {\freflnname\fancyrefdefaultspacing#1}

% listings

\lstset{%
  basicstyle=\footnotesize\ttfamily,
  aboveskip=0\baselineskip,
  belowskip=0\baselineskip,
  commentstyle=\scriptsize\itshape,
%  prebreak=\mbox{$\hookleftarrow$},
  breaklines,
  numberblanklines=false,
  numberstyle=\tiny\color{gray}, 
  numbersep=0pt,
  escapechar=`,  
  numberbychapter=false}
  
\lstdefinestyle{floating}
 {frame=lines,
  float=hptb,
  captionpos=b,
  abovecaptionskip=-0pt}

% context traits listings
\lstdefinestyle{py}
 {language=Python,
  showstringspaces=false,
  keywordstyle=\ttfamily\bfseries,
  tabsize=2,
  style=floating,
  belowskip=-0\baselineskip,
  aboveskip=-0\baselineskip,
  morekeywords={}
}

%context traits environment    
\lstnewenvironment{python}[1][]
 {\lstset{style=py,#1}}{}  

 % Context Traits in line source-code
\newcommand{\spy}[1]{\lstinline[style=py]{#1}}


%% Equations
\usepackage{tikz}
\usepackage[export]{adjustbox}
\usetikzlibrary{positioning,shadows,trees,mindmap}
\usetikzlibrary{arrows, shapes, backgrounds}
\usetikzlibrary{decorations.pathreplacing}
\usetikzlibrary{shapes.misc}
\usepackage[edges]{forest}
\usetikzlibrary{arrows.meta}
\colorlet{linecol}{black!75}
\usepackage{xkcdcolors} % xkcd colors
\usetikzlibrary{tikzmark}
\usetikzlibrary{calc}
% Commands for Highlighting text -- non tikz method
\newcommand{\highlight}[2]{\colorbox{#1!17}{#2}}
%\newcommand{\highlight}[2]{\colorbox{#1!17}{$#2$}}
\newcommand{\highlightdark}[2]{\colorbox{#1!47}{#2}}

  
%----[ Commands ]---


%Latins
\newcommand{\eg}{\emph{e.g.,}\xspace}
\newcommand{\ie}{\emph{i.e.,}\xspace}
\newcommand{\cf}{\emph{cf.}\xspace}

% Sets

\renewcommand{\emptyset}{\varnothing} % Redefine LaTeX version with AMS version

% Commands

\newcommand{\adaptiverl}{\textsc{morphin}\xspace}
\newcommand{\lrate}[1]{{\color{NavyBlue} $#1$}}

%comments
% xcolor
\definecolor{author}{rgb}{.5, .5, .5}
\definecolor{comment}{rgb}{.1, .0, .9}
\definecolor{note}{rgb}{.9, .4, .0}
\definecolor{idea}{rgb}{.1, .7, .0}
\definecolor{missing}{rgb}{.9, .1, .0}
\definecolor{OliveGreen}{rgb}{0,0.6,0.3}
\definecolor{Bittersweet}{rgb}{0.996,0.435,0.369}
\definecolor{RoyalBlue}{rgb}{0.25,0.41,0.88}
\definecolor{NavyBlue}{rgb}{0.0,0.4,0.8}
\definecolor{Mulberry}{rgb}{0.77,0.29,0.54}



\newcommand{\authorcomment}[3][comment]
  {\ifdraft{\noindent
      \fbox{\footnotesize\textcolor{author}{\textsc{#2}}}
      \textcolor{#1}{\textsl{#3}}}{}}


%----[ Hyphenation ]---

\hyphenation
  {a-vail-a-bil-i-ty
   im-ple-men-ta-tion
   caus-al-ly
   con-struct
   par-a-digm
   pro-gra-mming}

\makeatother

\endinput


% $Id: acronyms.tex  $
% !TEX root = main.tex

\acrodef{AI}{Artificial Intelligence}
\acrodef{API}{Application Programming Interface}
\acrodef{AOP}{Aspect-oriented Programming}
\acrodef{CAS}{Collective Adaptive Systems}
\acrodef{CONRL}[ConRL]{Constructivist Reinforcement Learning}
\acrodef{DSL}{Domain-Specific Language}
\acrodef{DNN}{Deep Neural Networks}
\acrodef{DQN}{Deep Q-learning}
\acrodef{LOC}{Lines of Code}
\acrodef{ML}{Machine Learning}
\acrodef{MDP}{Markov Decision Process}
\acrodef{MNC}{Multiply-Nested Container}
\acrodef{OOP}{Object-Oriented Programming}
\acrodef{PCA}{Principal Component Analysis}
\acrodef{PDG}{Program Dependence Graph}
\acrodef{RL}{Reinforcement Learning}



%---[ Acronyms Plurals Special Forms] ---


\newcommand{\acResetNonTrivial}
  {\acresetall
   \acused{CPU}
   \acused{API}
   \acused{LAN}
   \acused{SMT}
   \acused{GUI}}

\acResetNonTrivial


\endinput


\begin{document}

\title{Adapting the Behavior of Reinforcement Learning Agents to Changing Action Spaces and Reward Functions}

%\author{
%\IEEEauthorblockN{Raul de la Rosa}
%\IEEEauthorblockA{%\textit{ Department of Computer Science and Technology} \\
%\textit{Universidad de los Andes}\\
%Bogot\'a, Colombia \\
%c.delarosap@uniandes.edu.co}
%\and
%\IEEEauthorblockN{Ivana Dusparic}
%\IEEEauthorblockA{%\textit{School of Computer Science and Statistics} \\
%\textit{Trinity College Dublin}\\
%Dublin, Ireland \\
%ivana.dusparic@tcd.ie}
%\and
%\IEEEauthorblockN{Nicol\'as Cardozo}
%\IEEEauthorblockA{%\textit{Systems and Computing Engineering Department}\\
%\textit{Universidad de los Andes}\\
%Bogot\'a, Colombia \\
%n.cardozo@uniandes.edu.co}
%}
\author{
\IEEEauthorblockN{Author template}
\IEEEauthorblockA{%\textit{ Department of Computer Science and Technology} \\
\textit{University}\\
City, Country \\
university-email}
}

\maketitle




\begin{abstract}
In Reinforcement Learning, agents learn to solve specific tasks through the 
exploration of the environment in which they execute. Agents' behavior is
specialized towards a goal, gathered from the environment, according to the 
agent's experience from the execution of its actions. While effective in optimizing the behavior for a 
goal, the performance of agent's rapidly decreases when the environment conditions change.
This paper introduces \adaptiverl, a self-adaptive Q-learning agent that enables on-the-fly adaptation 
without full retraining. \adaptiverl leverages proactive environment monitoring and concept-drift 
detection to trigger dynamic adjustment of learning and exploration parameters only when needed, 
while preserving prior policy knowledge to mitigate catastrophic forgetting. Additionally, \adaptiverl 
supports seamless incorporation of new actions into the agent’s action space. We validate our approach 
on a standard RL benchmark with shifting reward functions and the introduction of new action. 
Additionally we validate our approach in a realistic traffic signal control application to manage road 
intersections. The results demonstrate the proposed approach evidences superior convergence 
speed, resource efficiency, and continuous adaptation to evolving conditions, with respect to the 
baseline.
\end{abstract}



\begin{IEEEkeywords}
Reinforcement Learning,  
Continual Reinforcement Learning,  
Q-learning,  
Concept Drift Detection,  
Adaptive Systems,  
Traffic Signal Control
\end{IEEEkeywords}

% $Id: introduction.tex $
% !TEX root = main.tex

\section{Introduction}
\label{sec:introduction}

\acf{RL} is a subset of \ac{ML} techniques in which an agent learns by interacting with an environment through trial and error. The agent receives scalar rewards or punishments based on its actions, aiming to maximize cumulative long-term rewards~\cite{sutton98}. Traditional RL algorithms, such as Q-Learning, assume stationary \acp{MDP}, where transition probabilities and reward functions are constant over time~\cite{meta-rl-traffic}. However, this assumption restricts RL's effectiveness in real-world environments, which often exhibit non-stationary characteristics like evolving state distributions, varying reward dynamics, and changing action spaces~\cite{khetarpal2022continualreinforcementlearningreview}.

To overcome these limitations, our work adopts the \acf{CRL} paradigm, where agents continuously adapt to environmental changes~\cite{abel2023definitioncontinualreinforcementlearning}. Approaches such as meta-learning and transfer learning have shown promise within \ac{CRL}. Meta-learning allows an agent to \emph{learn how to learn}, enhancing adaptability in new contexts~\cite{beck2024surveymetareinforcementlearning}, while transfer learning leverages prior knowledge to accelerate learning in related tasks~\cite{chen2022transferredqlearning}.

This paper introduces \adaptiverl, a tabular RL algorithm based on Q-Learning that incorporates adaptive mechanisms inspired by \ac{CRL}. Our method effectively addresses non-stationary environments by continuously updating its learning strategy. 

Existing literature highlights RL's suitability for self-adaptive systems, given its capacity to dynamically adjust behavior in response to environmental feedback~\cite{HENRICHS2022106940}. However, real-world scenarios such as traffic control present substantial challenges due to inherent non-stationary dynamics~\cite{meta-rl-traffic}.

We validate \adaptiverl in a self-adaptive system context, specifically applied to traffic signal control at intersections, a scenario where signal phases continuously influence state transitions and rewards~\cite{meta-rl-traffic}. By demonstrating the effectiveness of \adaptiverl under these conditions, we underline its suitability for self-adaptive systems, particularly those relying on streaming data where concept drift may occurs, making efficient adaptation essential.

This paper is structured as follows: Section~\ref{sec:background} introduces theoretical foundations relevant to our implementation. Section~\ref{sec:related} reviews related work. Section~\ref{sec:implementation} presents the \adaptiverl algorithm. Section~\ref{sec:validation} describes our validation approach in traffic signal control and discusses results. Finally, Section~\ref{sec:conclusion} offers conclusions and future research directions.





\endinput


% $Id: conclusion.tex 
% !TEX root = main.tex

%%
\section{Background}
\label{sec:background}

%%%%
\subsection{\acl{RL}}

In \ac{RL}, agents learn to map environmental situations (environment states) to actions that maximize a numerical reward signal received from the environment in the long term~\cite{sutton18}. \ac{RL} problems are formulated as Markov Decision Processes (MDPs), defined by the tuple $\mathcal{M} = \langle \mathcal{S}, \mathcal{A}, P, R, \gamma \rangle$, where:
$\mathcal{S}$ is the state space, comprising all relevant states of the environment;
$\mathcal{A}$ is the action space, i.e., the set of all actions the agent can perform to affect the environment;
$P(s_{t+1} \mid s_t, a_t)$ is the transition probability from state {\color{purple}$s_t$} to {\color{purple}$s_{t+1}$} under action {\color{purple}$a_t$};
$R(s_t, a_t)$ is the reward function, providing the numerical signal ($r$) that encodes the positive or negative impact of taking action {\color{purple}$a_t$} in state {\color{purple}$s_t$} at each execution step $t$; and $\gamma \in [0, 1]$ is the discount factor, which determines the importance of future rewards.

In stationary settings, $P$ and $R$ remain fixed over time, and the agent’s objective is to learn a policy $\pi: \mathcal{S} \to \mathcal{A}$ that maximizes the expected discounted return. However, real-world environments often violate this stationarity assumption: transition dynamics and/or reward functions may shift due to evolving conditions. We model such scenarios as \emph{non-stationary} MDPs, represented by a sequence
\[
\{\mathcal{M}_t\}_{t=1}^\infty
\quad\text{with}\quad
\mathcal{M}_t = \bigl\langle \mathcal{S}, \mathcal{A}_t, P_t, R_t, \gamma \bigr\rangle,
\]
where $P_t$ and/or $R_t$ change at unknown time steps $t$ (``concept drifts''). In this work, we further consider the novel scenario where the agent's action space $\mathcal{A}$ may also change over time. Detecting and adapting to these drifts is central to \ac{CRL}~\cite{khetarpal2022continualreinforcementlearningreview,abel2023definitioncontinualreinforcementlearning}, which treats learning as an ongoing process across a continually changing sequence of tasks.

Q-learning~\cite{watkins92} is a widely used model-free approach in \ac{RL}. The long-term quality of an action performed at a given state is computed iteratively in a series of steps and is represented by a Q value,
$\mathit{Q(s,a)}$.
Formally, each execution step $t$ captures information from the environment and maps it to a state
{\color{purple}$s_t$} $\in \mathcal{S}$ in its state space. It then selects an
action {\color{purple}$a_t$} $\in \mathcal{A}$ from its action space and executes it. The agent
receives a reward {\color{Bittersweet}$r_t$} from the environment when it moves to the next state
$s_{t+1} \in \mathcal{S}$. The reward is used to update the optimality of performing the
action {\color{Mulberry}$a_t$} at state {\color{purple}$s_t$}. The agent's goal is to learn
a policy (\i.e. the best-fit action for each state) that maximizes the reward of the
long-run behavior. The learning rate \lrate{\alpha} determines how much new experiences 
overwrite previously learned experiences, and the discount factor {\color{RoyalBlue} $\gamma$} 
determines how much future rewards are discounted so that agents prioritize immediate actions and 
can plan the best long-term actions. At each time step $t$, the Q value of an action 
{\color{purple}$a_{t+1}$} taken in state {\color{purple}$s_{t+1}$}, $Q(s_{t+1}, a_{t+1})$, is updated by 
the Bellman learning equation as follows:

\vspace{1em}
\input{equations/bellman}
\vspace{1em}

While Q-learning converges under stationary conditions (with appropriate decay of \lrate{\alpha}), it 
can struggle when $P_t$ or $R_t$ change over time. Our work builds on this foundation by 
incorporating online concept-drift detection and adaptive updates, enabling the agent to remain 
effective in non-stationary MDPs. We focus particularly on shifts in the reward function $R_t$, which 
may also induce changes in the action space $\mathcal{A}_t$.


%%%%
\subsection{Motivating example}
\label{sec:motivation}


To motivate the adaptation of agent's behavior to changing goals and the acquisition of new actions, 
we use two scenarios of the Gridworld benchmark as a running example.

Gridworld consists of a rectangular matrix ($9\times 9$ in our case). In Gridworld, every cell is 
associated with a reward of $-1$, except for a single goal state that has a reward of $+100$ as shown 
in the left-hand side of \fref{fig:r-change}. The agent begins each episode at the grid's center and may 
move in four directions: \spy{up}, \spy{down}, \spy{left}, or \spy{right}. Initially, the goal is placed in the 
top-left corner.  

In the first scenario, we allow the goal state of the environment to change. Such changes are unknown 
to the agents beforehand~\cite{cardozo21} and maybe due to the reallocation of 
objectives~\cite{khetarpal2022continualreinforcementlearningreview} (\eg reorganizing products in a 
warehouse for robotic fulfillment systems), or the shift to new objectives~\cite{florensa18} (\eg UAV drones 
changing their mission due to flight plan changes, or changing in leadership for flight formations).

\begin{figure*}[hptb]
    \centering
    \includegraphics[width=0.9\textwidth]{figures/rewards_change}
    \caption{Non-Stationary Gridworld: Concept drift is induced by relocating the reward. The agent starts at the center and must reach the goal, which alternates between the top-left and bottom-right corners every 300 episodes.}
    \label{fig:r-change}
\end{figure*}

Additionally, we consider a scenario of adding walls to the Gridworld, blocking the agent's movement 
according to its initial set of actions (\fref{fig:q-value-comp2}). For such scenario, we enable the agent 
to  extended its action space by acquiring new actions that provide additional capabilities to the agent 
(\ie jumping). Such situations are common in robotics for modular/evolving 
robots~\cite{eiben20evolving,miras20environmental}, or swarm behavior~\cite{schranz20swarm} where 
robots acquire new capabilities due to the physical attachment of parts, the possibility of combining 
actions with nearby robots, or the transfer of responsibilities/behavior between robots. Whichever the 
case may be, the agent is expected to adapt to these changes and efficiently adjust its policy using 
the newly acquired behavior.


\endinput


% $Id: conclusion.tex 
% !TEX root = main.tex

%%
\section{Implementation}
\label{sec:implementation}


Code repo \url{https://github.com/rulas99/rl_uniandes}

\endinput


% $Id: validation.tex 
% !TEX root = main.tex

%%
\section{Validation}
\label{sec:validation}

To evaluate the effectiveness of \adaptiverl, we conduct experiments in two distinct scenarios. First, we use the canonical Gridworld benchmark to provide a clear proof-of-concept and a running example of the agent's internal mechanisms. Second, we use a more complex traffic-signal control simulation to validate the agent's adaptability in a scenario representative of real-world self-adaptive systems.

%%%%
\subsection{Evaluation Scenarios}

%%%%%%
\subsubsection{Gridworld}
The Gridworld environment is a $9 \times 9$ grid, as described in \fref{sec:motivation}. We evaluate \adaptiverl under two distinct conditions within this environment, executing 1,000 independent runs for each to ensure statistical significance.
\begin{itemize}
    \item Changing Goals: The agent is run for 1,500 episodes. Every 300 episodes, the high-reward goal state is swapped between two diagonally opposite corners. The agent must detect this change and adapt its policy to the new goal.
    \item Expanding Action Space: In a separate experiment of 400 episodes, the agent is given a new action---the ability to ``jump'' over walls---at episode 300. The agent must incorporate this new action to find a more optimal path.
\end{itemize}

%%%%%%
\subsubsection{Traffic-Signal Control}
Intelligent traffic management is a canonical application of self-adaptive systems~\cite{HENRICHS2022106940}, where non-stationarity is a primary challenge~\cite{meta-rl-traffic}. We model a two-lane intersection using a custom environment, \texttt{TrafficEnv}, which extends OpenAI's Gym library~\cite{gymlib}.

The environment's Markov Decision Process (MDP) is defined as follows:
\begin{itemize}
    \item State ($S$): The state $s_t = (c_1, c_2)$ represents the number of queued vehicles in lane C1 (vertical) and C2 (horizontal), respectively. Queue lengths are bounded in the interval $[0, \mathit{max\_state}]$.
    \item Actions ($A$): An action corresponds to selecting a signal phase, defined by a tuple of service capacities (vehicles cleared per step). Initially, the agent has three actions: $\{(5,2), (2,5), (3,3)\}$.
    \item Transition Function ($P$): At each step, a sequence of events unfolds: 1) Service is applied to lane C1, reducing its queue. 2) New vehicles arrive at lane C2, governed by a Poisson distribution with rate $\lambda_2$. 3) Service is applied to lane C2. 4) New vehicles arrive at lane C1, governed by a Poisson distribution with rate $\lambda_1$.
    \item Reward Function ($R$): The reward function is designed to penalize both congestion and inefficient service. It is defined as $R_t = r_t - \text{penalty}_t$. The congestion cost, $r_t$, prioritizes the more congested lane:
    \[
    r_t = 
    \begin{cases}
    -(2c_1 + c_2)\quad &\text{if } c_1 > 7 \land c_1 > c_2,\\
    -(c_1 + 2c_2)\quad &\text{if } c_2 > 7 \land c_2 > c_1,\\
    -(c_1 + c_2)\quad &\text{otherwise}
    \end{cases}
    \]
    The service penalty, $\text{penalty}_t = 3 \times (\text{waste}_{C1} + \text{waste}_{C2})$, discourages allocating excessive green time to lanes with few vehicles.
\end{itemize}

Concept drift is induced by changing the vehicle arrival rates ($\lambda_1, \lambda_2$) at predefined episodes. When a drift is detected, the agent's action space is expanded with two new, more aggressive signal phases, $(7,3)$ and $(3,7)$, to provide finer control under heavy congestion. Existing actions are retained. The variations in $\lambda$ simulate realistic traffic patterns where congestion levels fluctuate due to temporal factors such as time of day, weather conditions, or special events. For instance, during morning rush hour, $\lambda_1$ might increase as the vertical corridor experiences heavy commuter traffic, while during late evening hours, $\lambda_2$ might increase as the horizontal route handles nightlife and service vehicle traffic.

%%%
\subsection{Experimental Setting}
Experiments are run on an Intel Core i5 CPU with 64GB RAM, using Python 3.12 and Gym 0.26.2. We compare \adaptiverl against a standard Q-learning baseline.
\begin{itemize}
  \item \textbf{Standard Q-learning (Baseline):} Uses a fixed learning rate $\alpha=0.1$ and an exploration rate $\varepsilon$ that decays exponentially from 0.9 to 0.01 without resets.
  \item \textbf{\adaptiverl:} Employs the PH-test to detect drifts and trigger an exploration reset ($\varepsilon \to 1$). It uses a dynamic learning rate $\alpha^*$ modulated by the TD-error, as described in \fref{sec:implementation}. The hyperparameters for the experiments (e.g., $k=5, \delta=0.5, H=300$) were determined empirically to suit the reward scale of each environment.
\end{itemize}

%%%%
\subsection{Evaluation Results}

%%%%%%
\subsubsection{Gridworld}
In the Gridworld scenario, \adaptiverl demonstrates superior adaptation to environmental changes compared to the baseline. \fref{fig:q-value-comp} shows the Q-value heatmaps after 1,500 episodes of goal switching. The \adaptiverl agent (left) retains high Q-values for both the current goal location and previously learned goal locations. This preservation of knowledge prevents catastrophic forgetting. In contrast, the standard Q-learning agent (right) overwrites its Q-values with each goal change, effectively forgetting the path to previous goals.

\begin{figure*}[hptb]
    \centering
    \includegraphics[width=0.9\textwidth]{figures/q_map_comp.png}
    \caption{Comparison of Q-value heatmaps after 1,500 episodes in the non-stationary Gridworld. (Left) \adaptiverl preserves high Q-values for both the initial goal (top-left) and subsequent goals, demonstrating knowledge retention. (Right) Standard Q-learning overwrites past knowledge, showing high values only for the most recent goal policy.}
    \label{fig:q-value-comp}
\end{figure*}

This adaptability translates to significant gains in learning efficiency. As shown in \fref{tab:gridworld-table}, \adaptiverl requires substantially fewer steps to converge after a drift. It needs $1.9\times$ fewer steps after the first goal change and converges in $1.7\times$ fewer total steps over the 1,500-episode run. The baseline agent fails to re-converge within the allotted episodes after the first change, highlighting its brittleness to non-stationarity.

\begin{table}[h]
\centering
\caption{Average steps to convergence after change and total steps over 1,500 episodes in the Gridworld scenario (1,000 runs). Dashes (--) indicate failure to converge within the 300-episode interval.}
\label{tab:gridworld-table}
\resizebox{\textwidth}{!}{
\begin{tabular}{l | c | c | c | c | c }
\toprule
\textbf{Agent} & \textbf{1st Change} & \textbf{2nd Change} & \textbf{3rd Change} & \textbf{4th Change} & \textbf{Total Steps} \\
\midrule
Q-Learning  & 256.40 $\pm$ 4.35\% & -- & -- & -- & 40,683.74 $\pm$ 1.33\% \\
\textbf{\adaptiverl} & 135.81 $\pm$ 9.31\% & 475.17 $\pm$ 5.80\% & 767.81 $\pm$ 3.13\% & 1,069.74 $\pm$ 2.45\% & 23,292.07 $\pm$ 6.33\% \\
\bottomrule
\end{tabular}
}
\end{table}

Furthermore, \adaptiverl effectively adapts to an expanded action space. \fref{fig:q-value-comp2} shows that when the ``jump'' action is introduced, \adaptiverl (left) successfully incorporates it to find a more optimal path (reducing steps from 14 to 4). The standard agent (right) struggles to leverage the new action in states where an old policy is already established, resulting in suboptimal performance.

\begin{figure*}[hptb]
    \centering
    \includegraphics[width=0.9\textwidth]{figures/q_map_comp2.png}
    \caption{Comparison of Q-value heatmaps after 400 episodes with an expanded action space. (Left) \adaptiverl effectively learns to use the new ``jump'' actions (indicated by double arrows) to create a more optimal policy. (Right) Standard Q-learning fails to integrate the new action effectively, remaining in a suboptimal policy.}
    \label{fig:q-value-comp2}
\end{figure*}

%%%%%%
\subsubsection{Traffic-Signal Control}
In the traffic-signal control scenario, \adaptiverl again demonstrates robust adaptation. \fref{fig:traffic-learning-curve} shows the cumulative mean reward for both agents over 10,000 episodes. At episode 3,000, a concept drift is induced by increasing traffic congestion. The PH-test in \adaptiverl detects this change promptly (green line), triggering an exploration reset and the introduction of new actions. Consequently, \adaptiverl's performance recovers rapidly, significantly outperforming the baseline Q-learning agent, which suffers a prolonged performance drop.

However, the results also highlight a limitation of the PH-test. At episode 8,000, a second drift occurs where traffic rates are lowered. This change is not detected by \adaptiverl because the resulting reward distribution is a subset of the rewards experienced during the initial, low-congestion phase. As the new rewards are not sufficiently novel to exceed the PH-test's sensitivity threshold, no adaptation is triggered. Both agents' performance improves as the environment becomes easier, but \adaptiverl's failure to detect the change underscores the challenge of selecting robust drift detection parameters.

Despite this, the case study confirms that \adaptiverl can effectively: react to detected non-stationary conditions, seamlessly incorporate new actions (signal phases) to manage new conditions, and leverage a dynamic reward function to promote resource-efficient policies. This enables a form of continuous, automated adjustment to traffic patterns that is impractical with manual retuning.

\begin{figure*}[hptb]
    \centering
    \includegraphics[width=0.9\textwidth]{figures/traffic_learning_curve.png}
    \caption{Learning performance for traffic-signal control under non-stationary congestion. After the first drift (episode 3,000), \adaptiverl detects the change and rapidly recovers performance by leveraging an expanded action set. Traditional Q-learning suffers a severe and prolonged performance degradation. The second drift (episode 8,000) is not detected by the PH-test.}
    \label{fig:traffic-learning-curve}
\end{figure*}

\endinput
% $Id: conclusion.tex 
% !TEX root = main.tex

%%
\section{Related Work}
\label{sec:related}

\ac{RL} inherently deals with the critical trade-off between exploration and exploitation, which 
significantly influences the performance of the learned policy. An agent must explore sufficiently to 
avoid settling for suboptimal solutions, yet excessive exploration can lead to inefficient training. 
Hence, determining an optimal balance between these two strategies is crucial for achieving 
high-quality solutions~\cite{sutton18}.

Many adaptive strategies have been proposed to manage this balance dynamically.~\citet{tokic2010} 
and~\citet{mignon2017adaptive} introduce adaptive implementations of the classic 
$\varepsilon$-greedy policy, highlighting the effectiveness of dynamically adjusting $\varepsilon$ values 
rather than maintaining them statically.~\citet{mignon2017adaptive} demonstrate how an adaptive 
approach enhances performance in both stationary and non-stationary environments by employing 
the PH-test for detecting environmental concept drifts.

Building on the exploration-exploitation dilemma, \citet{norman2024firstexploreexploitmetalearningsolve} 
propose First-Explore, a meta-\ac{RL} approach utilizing distinct policies dedicated to exploration 
and exploitation. Unlike conventional methods that directly optimize cumulative rewards, First-Explore 
trains these two separate policies and later combines them to form an inference policy that strategically 
explores initially for $k$ episodes, sacrificing immediate rewards for greater cumulative future gains. 
This strategy specifically addresses the limitations faced by existing methods that tend to prematurely 
converge to suboptimal solutions due to inadequate early exploration.

Hyperparameter optimization, particularly the learning rate, also plays a crucial role in training 
effective \ac{RL} models. A well-known example is the Adam 
optimizer~\cite{kingma2017adammethodstochasticoptimization}, which dynamically adjusts the learning 
rate based on past gradient information to facilitate more efficient training convergence. Extending 
this line of research, \citet{dynamicrlalpha} propose a dynamic learning rate approach tailored for deep 
\ac{RL} scenarios. Their method adaptively selects optimal learning rates at different training stages, 
demonstrating substantial performance improvements by considering the non-stationarity inherent to 
\ac{RL} tasks.

Transfer learning strategies, like the Transferred Q-Learning~\cite{chen2022transferredqlearning}, 
have demonstrated improvements in convergence rates. By reusing previous knowledge from similar 
tasks, showing that these methods effectively accelerate the learning process.

The methods mentioned above—focused on the adaptivity of hyperparameters and the reuse of prior knowledge—closely align with the concept of \ac{CRL}~\cite{khetarpal2022continualreinforcementlearningreview}.
\citet{abel2023definitioncontinualreinforcementlearning} define \ac{CRL} as a setting in which “the best agents never stop learning,” contrasting this with traditional \ac{RL}, which typically treats learning as the identification of a static solution rather than a process of continuous adaptation.
Recent work has begun to explore \ac{CRL} specifically through the lens of Q-learning.
\citet{Bagus2022} provide a systematic empirical studies of Continual Q-learning, using a decomposition of the original task into overlapping but non-contradictory sub-tasks to evaluate the effectiveness of continual learning mechanisms.
Another approach for Continual Q-Learning comes from \citet{araujo2020controladaptiveqlearning}, who propose Adaptive Q-Learning (AQL) and its single-partition variants SPAQL and SPAQL-TS. These algorithms dynamically refine state–action partitions during training and demonstrate strong sample efficiency in continuous control problems such as CartPole, without relying on fixed descretization.

Various studies have shown \ac{RL} as a potent tool for self-adaptive systems in real-world 
applications~\cite{HENRICHS2022106940}. Notable examples of \ac{RL} implementations 
include the use of adaptive $\varepsilon$-greedy policies to dynamically adjust exploration-exploitation 
trade-offs for \ac{IOT} security in edge computing~\cite{iotdynamicrl}, the integration of \ac{RL} 
with active learning and concept drift detection mechanisms (like PH-Test) for network monitoring to 
effectively identify potential threats~\cite{networkdynamicrl}, or the use of \ac{RL} macro actions to 
continuously learn adaptation strategies~\cite{cardozo23}.

Our approach unifies and extends three key research directions identified in the state of the art: 
\begin{enumerate*}[label=(\arabic*)] 
\item decoupling exploration and exploitation through specialized policies; 
\item online adaptation of hyperparameters based on performance feedback; and 
\item proactive environment monitoring and concept drift detection to trigger adaptive responses. 
\end{enumerate*}
Uniquely, \adaptiverl integrates these mechanisms within a \ac{CRL} framework that responds to detected drifts and action-space expansions, resetting exploration and updating hyperparameters only when necessary. This coordinated strategy enables continual adaptation without excessive retraining. Furthermore, addressing challenges such as those highlighted by \citet{Bagus2022}, our method mitigates catastrophic forgetting by preserving and reusing prior policy knowledge across evolving configurations, ensuring accelerated convergence even in the presence of overlapping contradictory subtasks.  

\endinput

% $Id: conclusion.tex 
% !TEX root = main.tex

%%
\section{Conclusion and Future Work}
\label{sec:conclusion}

This paper introduced \adaptiverl, a self-adaptive framework for tabular Q-learning agents operating in non-stationary environments. We specifically addressed the challenge of agents adapting to simultaneous changes in their goals (reward functions) and their available capabilities (action-space expansions). Our approach integrates concept drift detection using the PH-test with dynamic adjustments to the exploration ($\varepsilon$) and learning ($\alpha$) rates. Experimental results in both a Gridworld benchmark and a traffic control simulation demonstrate that this coordinated strategy enables agents to adapt more effectively than a standard Q-learning baseline, achieving a performance increase of up to $1.7\times$ in learning efficiency while successfully reducing catastrophic forgetting effects.

Through this synthesis of established techniques into a unified framework, \adaptiverl demonstrates how a lightweight, model-free agent can achieve robust continual learning. By preserving and adapting a single Q-table, it effectively enables knowledge reuse when faced with contradictory environmental changes, a key challenge highlighted by~\citet{Bagus2022}, without requiring full retraining or multiple context models. Our work thus presents a practical and resource-efficient method for developing self-adaptive systems capable of real-time learning in dynamic environments.

This work represents an initial validation, and several avenues for future work are evident. The limitations observed in our experiments, such as the failure of the PH-test to detect certain drifts and the need for empirical tuning of hyperparameters, point to clear directions for improvement. Future work should therefore focus on:
\begin{enumerate}
    \item Generalization and Scalability: Extending the core principles of \adaptiverl from tabular methods to deep \ac{RL} architectures to handle high-dimensional state spaces. This would also involve investigating the use of memory-based plasticity to enhance knowledge transfer across tasks.
    \item Robustness and Empirical Analysis: Conducting a rigorous sensitivity analysis of the introduced hyperparameters, particularly the TD-error sensitivity ($k$) and $H$ threshold for the PH-test, to better understand their impact on performance. Furthermore, we plan to explore more robust drift detection methods to overcome the limitations of the PH-test observed in our traffic scenario, where drifts that result in subsets of known reward distributions can be missed. We also intend to extend the evaluation to include scenarios with action removal and other types of reward function changes, such as those examined by \citet{mignon2017adaptive}.
    \item Advanced Application Domains: Applying the framework to more complex distributed multi-agent systems (\eg bigger street network configuration). Finally, we plan to deploy \adaptiverl on resource-constrained edge devices for real-world applications in \ac{IOT} and robotics.
\end{enumerate}

\endinput

%\section*{Acknowledgment}

\printbibliography

\end{document}
