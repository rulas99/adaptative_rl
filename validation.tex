% $Id: conclusion.tex 
% !TEX root = main.tex

%%
\section{Validation}
\label{sec:validation}

This section presents our validation case studies and the results obtained from the execution of \adaptiverl with respect to the state-of-the-art. 
%%%%
\subsection{Scenarios}

The evaluation is situated in the domain of Intelligent Traffic Management, particularly in traffic signal 
control, which is an established application area for self-adaptive 
systems~\cite{HENRICHS2022106940}. meta-\ac{RL} has been used in traffic signal control in 
non-stationarity scenarios that dynamically switch reward functions based on traffic flow saturation 
levels. As a baseline of our work we use a \acf{DQN} method to optimize traffic signal control, achieving 
substantial improvements in terms of reduced waiting times and queue 
lengths~\cite{Swapno2024,MORENOMALO2024124178}.

Our scenario consists of a state-action space configuration based on the number of vehicles per lane, with phase durations as defined actions, thus contributing further to efficient and sustainable urban mobility.
\authorcomment[missing]{NC}{Scenario description}


%%%%
\subsection{Results}

In the experiments realized, for the base adaptive q-learning and its application over traffic lights controlling \authorcomment[idea]{NC}{Complete}



\endinput

