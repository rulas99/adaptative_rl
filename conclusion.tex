% $Id: conclusion.tex 
% !TEX root = main.tex

%%
\section{Conclusion and Future Work}
\label{sec:conclusion}

This paper presents \adaptiverl, a self-adaptive approach to enable Q-learning agents to effectively 
detect and adapt to non-stationary environments. In particular, with this work we address the problem 
of agents' adaptation to changes in their objectives or goal states, as well as changes in their action 
space by incorporating new actions.
The adaptive capabilities of \adaptiverl agents are aligned with the principles of self-adaptive systems, 
enabling the agent to autonomously adjust its behavior in response to evolving environmental 
conditions.
Agent adaptation to changing goals and actions is realized by leveraging concept drift detection, the 
introduction of the dynamic learning rate, and the dynamic adjustment of the exploration hyperparameter. 
\adaptiverl proves effective in adapting agents' behavior to the environment, resulting in a sustained 
performance, in contrast of the baseline Q-learning agents, which observe a significant performance 
drop upon environment changes. 

The adaptability strategy proposed with \adaptiverl is in line with the ideas of 
\ac{CRL}~\cite{abel2023definitioncontinualreinforcementlearning}, treating learning as a continuous 
process rather than a one-time optimization. Notably, \adaptiverl addresses one of the major 
challenges put forward in \ac{CRL}: the problem of catastrophic forgetting. \adaptiverl retains prior 
policy knowledge while adapting to new configurations, enabling knowledge reuse and accelerating 
convergence as a consequence, even when faced with overlapping contradictory 
subtasks~\citet{Bagus2022}. The observed performance increase when using \adaptiverl is between 
$1.7\times$ and $1.9\times$.

We validate the effectiveness of \adaptiverl in two application domains, a proof of concept \ac{RL} 
benchmark, and a realistic traffic signal control scenarios. In both scenarios, the \adaptiverl agent 
successfully adapts to the changing environment conditions, switching of goals or introduction of new 
actions in the first case, and changing traffic patterns and performance requirements in the second 
case. In both cases, \adaptiverl evidences a performance improvement in contrast to the baseline 
Q-learning implementation.
The introduction of new actions (be that as specific signals as in Gridworld, or in response to 
drift detection in the traffic signal application), \adaptiverl demonstrated its ability to acquire new 
capabilities without discarding existing ones. This is particularly relevant in real-world traffic systems, 
where signal configurations can be updated in real time; equipping them with adaptive intelligence to 
allow the system for continuous optimization without the need for manual system retuning.
Finally, for the traffic signal control application, we recognize that the use of the dynamic reward 
function, including penalties for over-serving, proved effective in promoting resource-efficient policies. 

In conclusion, \adaptiverl stands as a promising approach for self-adaptive systems requiring 
lightweight, real-time learning solutions in environments with limited computational resources.
\adaptiverl is the first tabular self-adaptive agent to seamlessly integrate concept drift detection, 
dynamic hyperparameter adaptation, and on-the-fly action-space expansion within a unified 
framework. This approach addresses a significant gap in existing methods and represents a step 
forward toward robust continual learning agents capable of operating in highly dynamic domains.

Our work present an initial evaluation an adaptation of \ac{RL} agents to changing environment 
conditions. As avenues of future work, we aim to generalize its core principles from tabular methods to 
deep learning architectures. This includes incorporating memory-based plasticity to support knowledge 
reuse across tasks and enabling online test-time adaptation, allowing agents to respond to new 
situations without requiring full retraining.

Moreover, we will extend the evaluation of our work to different application domains combining both 
goal changes and the introduction of new actions to increase the generalization of our results. Among 
the extension to the evaluation we envision an empirical evaluation of the introduced learning 
parameter $k$, the removal of agent actions, and the combination of environment changes under 
different contexts.

Finally, we plan to integrate \adaptiverl with complementary mechanisms, such as multi-agent coordination for distributed adaptive systems, and deploy it in resource-constrained edge devices for real-world applications including \ac{IOT} scenarios and autonomous robotics systems.


\endinput

